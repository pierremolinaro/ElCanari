%!TEX encoding = UTF-8 Unicode

\documentclass [a4paper, 11pt, openany] {book}
\usepackage{verbatim}
%\usepackage{geometry}                % See geometry.pdf to learn the layout options. There are lots.
%\geometry{a4paper}                   % ... or a4paper or a5paper or ... 
%\geometry{landscape}                % Activate for for rotated page geometry
%\usepackage[parfill]{parskip}    % Activate to begin paragraphs with an empty line rather than an indent

%-----------------------------------------------------------------------------------------------------------------------*
%                                                                                                                       *
%   E N C O D A G E    D E S    S O U R C E S     :     U T F 8                                                         *
%                                                                                                                       *
%-----------------------------------------------------------------------------------------------------------------------*

%--- Paquetage pour le codage des sources en UTF-8
\usepackage[utf8]{inputenc}

%--- Pour les lettres grecques dans le mode texte : \textmu, …
% https://texblog.org/2012/03/15/greek-letters-in-text-without-changing-to-math-mode/
\usepackage{textgreek}

%--- Latex demande ce paquetage pour mieux afficher le caractère "°" et \textquotesingle "'"
\usepackage{textcomp}

%--- Ce paquetage permet d'effectuer certaines césures, et ainsi d'éviter les messages "Overfull \hbox"
\usepackage[T1]{fontenc}

\usepackage{lmodern} % for French

\usepackage{filecontents}

%-----------------------------------------------------------------------------------------------------------------------*
%                                                                                                                       *
%   R É G L A G E S    « F R A N Ç A I S »                                                                              *
%                                                                                                                       *
%-----------------------------------------------------------------------------------------------------------------------*

%--- Paquetage pour imposer les réglages français
%\usepackage[francais]{babel}
\usepackage[frenchb]{babel}

%--- Contrôle de l'indentation et de la séparation des paragraphes
\setlength{\parindent}{0pt} 
%\setlength{\parskip}{1.2ex} % Reporté avant les chapitres

%--- Ajouter une séparation à la fin des itemize
\let\EndItemize\enditemize
\def\enditemize{\EndItemize\vspace{1.2ex}}

%-----------------------------------------------------------------------------------------------------------------------*
%                                                                                                                       *
%   M I S E    E N    P A G E                                                                                           *
%                                                                                                                       *
%-----------------------------------------------------------------------------------------------------------------------*

%--- Interligne 1,5 ligne
\usepackage{setspace}
\onehalfspacing

%--- Marge gauche : 2,8 cm ; le paramètre \hoffset contient cette valeur, moins 1 pouce
%    \hoffset = 2,8 cm - 2,54 cm = 0,26 cm
\setlength{\hoffset}{0.26 cm}

%--- Marges supplémentaires, différenciées pour les pages gauches et droites ; ici, aucune.
\setlength{\oddsidemargin }{0 cm}
\setlength{\evensidemargin}{0 cm}

%--- Largeur du texte
%    \textwidth = 210 mm - 28 mm - 28 mm = 15,4 cm
\setlength{\textwidth}{15.4 cm}

%--- Marge haute : 2,8 cm ; le paramètre \voffset contient cette valeur, moins 1 pouce
%    \voffset = 2,8 cm - 2,54 cm = 0,26 cm
\setlength{\voffset}{0.26 cm}

%--- Distance entre la marge haute et l'en-tête : 0 cm
\setlength{\topmargin}{0 cm}

%--- Hauteur de l'en-tête de chaque page : 1 cm
\setlength{\headheight}{1 cm}

%--- Distance entre l'en-tête de chaque page et le corps : 0,5 cm
\setlength{\headsep}{0.5 cm}

%--- Hauteur du corps
%    \textheight = 29,7 cm - 2,8 cm - 2,8 cm - 1,5 cm = 22,6 cm
\setlength{\textheight}{22.6 cm}

%-----------------------------------------------------------------------------------------------------------------------*
%                                                                                                                       *
%   T R A C E R    U N E    L I G N E    H O R I Z O N T A L E                                                          *
%                                                                                                                       *
%-----------------------------------------------------------------------------------------------------------------------*

\newcommand\ligne{\hrulefill}

%-----------------------------------------------------------------------------------------------------------------------*
%                                                                                                                       *
%   C H O I X    D E    L A    P O L I C E                                                                              *
%                                                                                                                       *
%-----------------------------------------------------------------------------------------------------------------------*

%---------------------------------------------------- Pour utiliser la police "Fourier"
%\usepackage{fouriernc}
%\usepackage[scaled=0.875]{helvet}

%\usepackage[scaled=0.8, default]{sourcesanspro}
\usepackage[scaled=0.9, default, semibold]{sourcecodepro}
\usepackage[default]{sourcesanspro}

%-----------------------------------------------------------------------------------------------------------------------*
%                                                                                                                       *
%   E X T E N S I O N S    P O U R    P R É S E N T E R    L E S    T A B L E A U X                                     *
%                                                                                                                       *
%-----------------------------------------------------------------------------------------------------------------------*

\usepackage{array}

%  http://en.wikibooks.org/wiki/LaTeX/Colors
\usepackage[dvipsnames, svgnames, table]{xcolor} % À placer avant \usepackage{listings}

%--- Ce paquetage permet de changer le style des légendes des tableaux et des figures (voir caption-eng.pdf) :
%  - l'étiquette est en italique gras
%  - le titre est en italique.
\usepackage[font=it, labelfont=bf]{caption}

%--- Par défaut, le paquetage nomme "Table" les tableaux. La commande
%   suivante impose le nom "Tableau"
% Voir http://fr.wikibooks.org/wiki/LaTeX/Éléments_flottants_et_figures
\addto\captionsfrench{\def\tablename{Tableau}}

%------------------------------------------------------------------------------------------ RÉFÉRENCES À UN TABLEAU
% La référence au tableau "nom-du-tableau" est définie par \labelTableau{nom-du-tableau}
\newcommand\labelTableau[1]{\label{tab:#1}}
% Latex autorise deux types d'appel à une référence \ref{tab:nom-du-tableau} et \pageref{tab:nom-du-tableau}

% \refTableau{}{nom-du-tableau} ---> "tableau x.y"   où x.y est le n° du tableau
\newcommand\refTableau[1]{\hyperref[tab:#1]{tableau \ref*{tab:#1}}}

% \refTableauSansPrefixe{}{nom-du-tableau} ---> "x.y"   où x.y est le n° du tableau
\newcommand\refTableauSansPrefixe[1]{\hyperref[tab:#1]{\ref*{tab:#1}}}

% \refTableauPage{}{nom-du-tableau} ---> "tableau x.y page n"   où x.y est le n° du tableau
\newcommand\refTableauPage[1]{\hyperref[tab:#1]{tableau \ref*{tab:#1} page \pageref{tab:#1}}}

% \refTableauPageSansPrefixe{}{nom-du-tableau} ---> "x.y page n"   où x.y est le n° du tableau
\newcommand\refTableauPageSansPrefixe[1]{\hyperref[tab:#1]{\ref*{tab:#1} page \pageref{tab:#1}}}

%-----------------------------------------------------------------------------------------------------------------------*
%                                                                                                                       *
%   T I K Z    -    P G F                                                                                               *
%                                                                                                                       *
%-----------------------------------------------------------------------------------------------------------------------*

\usepackage{tikz}
\usepackage{tkz-graph}
\usetikzlibrary{calc}
\usepackage{pgfplots}
\usetikzlibrary{arrows}
\usetikzlibrary{decorations}
\usetikzlibrary{decorations.pathmorphing}
\usetikzlibrary{shapes.callouts}
\usetikzlibrary{shapes.misc}
\usetikzlibrary{automata}
\usetikzlibrary{positioning}
\usepgflibrary{shapes.geometric}

%-----------------------------------------------------------------------------------------------------------------------*
%                                                                                                                       *
%   E N - T Ê T E S    E T    P I E D S    D E    P A G E S                                                             *
%                                                                                                                       *
%-----------------------------------------------------------------------------------------------------------------------*

% Grâce au package "fancyhdr"
% voir http://www.exomatik.net/U-Latex/Personnaliser#toc2
%      http://www.trustonme.net/didactels/250.html
\usepackage{fancyhdr}
\pagestyle{fancy}
%--- Numéro de page : à gauche pages paires, à droite pages impaires
\fancyhead[EL,OR]{\thepage}
%--- Nom de chapitre : à droite page paires
\fancyhead[ER]{\leftmark}
%--- Nom de section : à gauche page impaires
\fancyhead[OL]{\rightmark}
%--- Version : au milieu du pied de chaque page
\fancyfoot[C]{Calculs géométriques de ElCanari}
%--- filet en haut et en bas de chaque page
\renewcommand{\headrulewidth}{0.5 pt}
\renewcommand{\footrulewidth}{0.5 pt}

\renewcommand{\chaptermark}[1]{\markboth{\bsc{\chaptername~\thechapter{}.} #1}{}}
\renewcommand{\sectionmark}[1]{\markright{\bsc{\thesection{}.} #1}{}}

%-----------------------------------------------------------------------------------------------------------------------*
%                                                                                                                       *
%   G E S T I O N    D E    L ' I N D E X                                                                               *
%                                                                                                                       *
%-----------------------------------------------------------------------------------------------------------------------*

% http://www.cuk.ch/articles/4097
% http://www.tuteurs.ens.fr/logiciels/latex/makeindex.html
% http://linux.die.net/man/1/makeindex
%
% Attention ! Les deux commandes suivantes, ainsi que le \printindex placé plus bas ne
% sont pas suffisants pour construire l'index : il faut utiliser l'utilitaire "makeIndex"
% Voir le fichier de commande "build.command"
\usepackage{makeidx}
\makeindex

%-----------------------------------------------------------------------------------------------------------------------*
%                                                                                                                       *
%   T O C B I D I N D                                                                                                   *
%                                                                                                                       *
%-----------------------------------------------------------------------------------------------------------------------*

%    Pour faire figurer la liste des tableaux (et la table des matières)
%    dans la table des matières
\usepackage{tocbibind}

\setcounter{tocdepth}{3}

%\titlespacing{\chapter} {0pt} {*0} {*0} {}
%\titlespacing{\section} {4ex} {*0} {*0} {}
%\titlespacing{\subsection} {10ex} {*0} {*0} {}
%\titlespacing{\subsubsection} {\subsubsectskip} {*0} {*0} {}

%\setlength{\parindent}{50pt}
%\makeatletter
%\renewcommand\paragraph{\@startsection{paragraph}{4}{\z@}%
%                                    {3.25ex \@plus 1ex \@minus .2ex}%
%                                    {2.3ex \ at plus.2ex}%
%                                    {\normalfont\normalsize\bfseries}}
%\makeatother

%-----------------------------------------------------------------------------------------------------------------------*
%                                                                                                                       *
%   H Y P E R R E F                                                                                                     *
%                                                                                                                       *
%-----------------------------------------------------------------------------------------------------------------------*

%--- Pour les hyperliens, et le contrôle de la génération PDF 
\usepackage{hyperref}
\hypersetup{colorlinks=true}
\hypersetup{linkcolor=blue}
\hypersetup{breaklinks=true}

%-----------------------------------------------------------------------------------------------------------------------*
%                                                                                                                       *
%   R É F É R E N C E S                                                                                                 *
%                                                                                                                       *
%-----------------------------------------------------------------------------------------------------------------------*

\usepackage{subfig}

%---- Nécessaire pour \refFigure...
\usepackage{ifthen}

% Au lieu d'écrire \chapter{titre-chapitre}, on écrit \chapterLabel{titre-chapitre}{label-chapitre}
\newcommand \chapterLabel[2]{\chapter{#1}\label{chapter:#2}}

% \refChapter{label-chapter} ---> "chapitre n"
\newcommand \refChapter[1]{\hyperref[chapter:#1]{chapitre \ref*{chapter:#1}}}

\newcommand \refChapterPage[1]{\hyperref[chapter:#1]{chapitre \ref*{chapter:#1} page \pageref{chapter:#1}}}

% Au lieu d'écrire \section{titre-section}, on écrit \sectionLabel{titre-section}{label-section}
\newcommand \sectionLabel[2]{\section{#1}\label{sec:#2}}


% \refSectionPage{label-section} ---> "section x.y page n"   où x.y est le n° de la section
\newcommand\refSectionPage[1]{\hyperref[sec:#1]{section \ref*{sec:#1} page \pageref{sec:#1}}}

%------------------------------------------------------------------------------------------ RÉFÉRENCES À UNE SUB-SECTION
% Au lieu d'écrire \subsection{titre-section}, on écrit \subsectionLabel{titre-section}{label-subsection}
\newcommand \subsectionLabel[2]{\subsection{#1}\label{subsec:#2}}


% \refSubsectionPage{label-section} ---> "section x.y page n"   où x.y est le n° de la sub-section
\newcommand\refSubsectionPage[1]{\hyperref[subsec:#1]{section \ref*{subsec:#1} page \pageref{subsec:#1}}}

%------------------------------------------------------------------------------------------ RÉFÉRENCES À UNE SUB-SUB-SECTION
% Au lieu d'écrire \subsubsection{titre-section}, on écrit \subsubsectionLabel{titre-section}{label-subsubsection}
\newcommand \subsubsectionLabel[2]{\subsubsection{#1}\label{subsubsec:#2}}


% \refSubsubsectionPage{label-section} ---> "section x.y page n"   où x.y est le n° de la sub-section
\newcommand\refSubsubsectionPage[1]{\hyperref[subsubsec:#1]{section \ref*{subsubsec:#1} page \pageref{subsubsec:#1}}}

\newcommand\refSubsubsectionTitlePage[1]{\hyperref[subsubsec:#1]{section « \emph{\nameref*{subsubsec:#1}} » page \pageref{subsubsec:#1}}}

%------------------------------------------------------------------------------------------ RÉFÉRENCES À UNE FIGURE
% La référence au tableau "nom-de-la-figure" est définie par \labelFigure{nom-de-la-figure}
\newcommand\labelFigure[1]{\label{fig:#1}}
% Latex autorise deux types d'appel à une référence \ref{fig:nom-de-la-figure} et \pageref{fig:nom-de-la-figure}

% \refFigure{}{nom-de-la-figure}   ---> "figure x.y"   où x.y est le n° de la figure
% \refFigure{z}{nom-de-la-figure}  ---> "figure x.y.z" où x.y est le n° de la figure
\newcommand\refFigure[2]{\hyperref[fig:#2]{figure \ref*{fig:#2}{\ifthenelse{\equal{#1}{}}{}{.#1}}}}

% \refFigureSansPrefixe{}{nom-de-la-figure}   ---> "x.y"   où x.y est le n° de la figure
% \refFigureSansPrefixe{z}{nom-de-la-figure}  ---> "x.y.z" où x.y est le n° de la figure
\newcommand\refFigureSansPrefixe[2]{\hyperref[fig:#2]{\ref*{fig:#2}{\ifthenelse{\equal{#1}{}}{}{.#1}}}}

% \refFigurePage{}{nom-de-la-figure}   ---> "figure x.y page n"   où x.y est le n° de la figure
% \refFigurePage{z}{nom-de-la-figure}  ---> "figure x.y.z page n" où x.y est le n° de la figure
\newcommand\refFigurePage[2]{\hyperref[fig:#2]{figure \ref*{fig:#2}{\ifthenelse{\equal{#1}{}}{}{.#1}} page \pageref{fig:#2}}}

% \refFigurePageSansPrefixe{}{nom-de-la-figure}   ---> "x.y page n"   où x.y est le n° de la figure
% \refFigurePageSansPrefixe{z}{nom-de-la-figure}  ---> "x.y.z page n" où x.y est le n° de la figure
\newcommand\refFigurePageSansPrefixe[2]{\hyperref[fig:#2]{\ref*{fig:#2}{\ifthenelse{\equal{#1}{}}{}{.#1}} page \pageref{fig:#2}}}

%-----------------------------------------------------------------------------------------------------------------------*
%                                                                                                                       *
%   E X T E N S I O N S    P O U R    L ' É C R I T U R E    D E S     F O R M U L E S    M A T H É M A T I Q U E S     *
%                                                                                                                       *
%-----------------------------------------------------------------------------------------------------------------------*

\usepackage{graphicx}
\usepackage{amssymb}
\usepackage{epstopdf}
\DeclareGraphicsRule{.tif}{png}{.png}{`convert #1 `dirname #1`/`basename #1 .tif`.png}

%-----------------------------------------------------------------------------------------------------------------------*
%   A F F I C H A G E    D U    C O D E    S H E L L                                                                    *
%-----------------------------------------------------------------------------------------------------------------------*

\newcommand\tpp[1]{\colorbox{gray!12}{\ttfamily #1}}

%-----------------------------------------------------------------------------------------------------------------------*
%                                                                                                                       *
%   C O D E   S O U R C E    S W I F T                                                                                  *
%                                                                                                                       *
%-----------------------------------------------------------------------------------------------------------------------*

\usepackage{listings}

\lstdefinelanguage{swift}
{
  morekeywords={
    func,if,then,else,for,in,while,do,switch,case,default,where,break,continue,fallthrough,return,
    typealias,struct,class,enum,protocol,var,func,let,get,set,willSet,didSet,inout,init,deinit,extension,
    subscript,prefix,operator,infix,postfix,precedence,associativity,left,right,none,convenience,dynamic,
    final,lazy,mutating,nonmutating,optional,override,required,static,unowned,safe,weak,internal,
    private,public,is,as,self,unsafe,dynamicType,true,false,nil,Type,Protocol,import
  },
  morecomment=[l]{//}, % l is for line comment
  morecomment=[s]{/*}{*/}, % s is for start and end delimiter
  morestring=[b]" % defines that strings are enclosed in double quotes
}

\definecolor{keyword}{HTML}{BA2CA3}
\definecolor{string}{HTML}{D12F1B}
\definecolor{comment}{HTML}{008400}

\lstset{
  language=swift,
  basicstyle=\ttfamily\small,
  showstringspaces=false, % lets spaces in strings appear as real spaces
  columns=fixed,
  keepspaces=true,
  keywordstyle=\color{keyword},
  stringstyle=\color{string},
  commentstyle=\color{comment},
  frame=L,
}

%-----------------------------------------------------------------------------------------------------------------------*

\usepackage{mdframed}
\usepackage{amsmath,amsfonts,amssymb}

%-----------------------------------------------------------------------------------------------------------------------*
%                                                                                                                       *
%   D É B U T    D U    D O C U M E N T                                                                                 *
%                                                                                                                       *
%-----------------------------------------------------------------------------------------------------------------------*

\begin{document} 

%-----------------------------------------------------------------------------------------------------------------------*
%                                                                                                                       *
%   P A G E    D E    T I T R E                                                                                         *
%                                                                                                                       *
%-----------------------------------------------------------------------------------------------------------------------*


\title{\Huge\bf Algorithmes géométriques\\utilisés dans ElCanari}
\author{Pierre Molinaro}
\date \today 
\maketitle

%-----------------------------------------------------------------------------------------------------------------------*
%                                                                                                                       *
%   T A B L E    D E S    M A T I È R E S                                                                               *
%                                                                                                                       *
%-----------------------------------------------------------------------------------------------------------------------*

\tableofcontents

%-----------------------------------------------------------------------------------------------------------------------*
%                                                                                                                       *
%   L I S T E    D E S    T A B L E A U X                                                                               *
%                                                                                                                       *
%-----------------------------------------------------------------------------------------------------------------------*

\listoftables
\addtocontents{lot}{\protect\thispagestyle{empty}\protect\pagestyle{empty}}

%-----------------------------------------------------------------------------------------------------------------------*
%                                                                                                                       *
%   L I S T E    D E S    F I G U R E S                                                                                 *
%                                                                                                                       *
%-----------------------------------------------------------------------------------------------------------------------*

\listoffigures
\addtocontents{lof}{\protect\thispagestyle{empty}\protect\pagestyle{empty}}

%-----------------------------------------------------------------------------------------------------------------------*
%                                                                                                                       *
%   L E S    C H A P I T R E S                                                                                          *
%                                                                                                                       *
%-----------------------------------------------------------------------------------------------------------------------*

%--- Contrôle de la séparation des paragraphes
%    On met cette définition ici, sinon elle affecte la table des matières, la liste des tableaux, ...
\setlength{\parskip}{1.2ex}

%!TEX encoding = UTF-8 Unicode
%!TEX root = ../algos-geometriques.tex


\chapter{Points}

Dans ce chapitre, un point $P$ est défini par ses coordonnées $(P.x, P.y)$. Les fonctions sont définies par une extension de \texttt{CGPoint} dans le fichier \texttt{extension-CGPoint.swift}.


\section{Distance entre deux points}

Élementaire !


\begin{lstlisting}
extension CGPoint {
  static func distance (_ p1 : CGPoint, _ p2 : CGPoint) -> CGFloat {
    let dx = p1.x - p2.x
    let dy = p1.y - p2.y
    return sqrt (dx * dx + dy * dy)
  }
}
\end{lstlisting}




\section{Fonction \texttt{product}}

Cette fonction permet de savoir si un point $P_3$ est situé \emph{à droite} (comme dans la figure ci-dessous) ou \emph{à gauche} d'un segment $P_1P_2$. Cette fonction est fondamentale pour de nombreux calculs, comme par exemple l'intersection de deux rectangles.

\begin{center}
  \begin{tikzpicture}
    \draw[thick] (0, 0) -- (2, 1) ;
    \draw[fill] (0, 0) circle (2pt) ;
    \draw[fill] (2, 1) circle (2pt) ;
    \draw[left] (0, 0) node {$P_1$} ;
    \draw[right] (2, 1) node {$P_2$} ;
    \draw[fill] (3, 0.5) circle (2pt) ;
    \draw[right] (3, 0.5) node {$P_3$} ;
  \end{tikzpicture}
\end{center}

Pour cela, on calcule la composante verticale du produit vectoriel $\overrightarrow{P_1P_2} \wedge \overrightarrow{P_1P_3}$ :
\begin{itemize}
\item positive, le point $P_3$ est à gauche du segment $P_1P_2$ ;
\item négative, le point $P_3$ est à droite du segment $P_1P_2$ ;
\item nulle, le point $P_3$ est aligné avec le segment $P_1P_2$.
\end{itemize}


\begin{lstlisting}
extension CGPoint {
  static func product (_ p1 : CGPoint, _ p2 : CGPoint, _ p3 : CGPoint) -> CGFloat {
    let dx2 = p2.x - p1.x
    let dy2 = p2.y - p1.y
    let dx3 = p3.x - p1.x
    let dy3 = p3.y - p1.y
    return dx2 * dy3 - dx3 * dy2
  }
}
\end{lstlisting}




\section{Angle d'un vecteur avec l'horizontal}

Étant donnés deux points $P_1$ et $P_2$, il s'agit de déterminer l'angle $\alpha$ que fait le vecteur $\overrightarrow{P_1P_2}$ avec l'horizontale :

\begin{center}
  \begin{tikzpicture}
    \draw[thick] (0, 0) -- (2, 1) ;
    \draw (0, 0) -- (2, 0) ;
    \draw[fill] (0, 0) circle (2pt) ;
    \draw[fill] (2, 1) circle (2pt) ;
    \draw[left] (0, 0) node {$P_1$} ;
    \draw[right] (2, 1) node {$P_2$} ;
    \draw[->] (1,0) arc (0:26.57:1) ;
    \draw[right] (1, .25) node {$\alpha$} ;
  \end{tikzpicture}
\end{center}

La fonction \texttt{atan2} renvoie l'angle en radian, et gère tous les cas particuliers (voir le \emph{man} de cette fonction) :
\begin{itemize}
  \item \texttt{atan2(+-0, -0)} retourne $\pm\pi$ ;
  \item \texttt{atan2(+-0, +0)} retourne $\pm0$ ;
  \item \texttt{atan2(+-0, x)} retourne $\pm\pi$ pour $x < 0$ ;
  \item \texttt{atan2(+-0, x)} retourne $\pm0$ pour $x > 0$ ;
  \item \texttt{atan2(y, +-0)} retourne $+\pi/2$ pour $y > 0$ ;
  \item \texttt{atan2(y, +-0)} retourne $-\pi/2$ pour $y < 0$.
\end{itemize}



\begin{lstlisting}
extension CGPoint {
  static func angleInRadian (_ p1 : CGPoint, _ p2 : CGPoint) -> CGFloat {
    let width = p2.x - p1.x
    let height = p2.y - p1.y
    return atan2 (height, width) // Result in radian
  }
}
\end{lstlisting}

%!TEX encoding = UTF-8 Unicode
%!TEX root = ../algos-geometriques.tex


\chapterLabel{Rectangle horizontal}{rectangleHorizontal}

Par rectangle « horizontal », on entend un rectangle dont les côtés sont parallèles aux axes. Un tel rectangle est décrit par le type \texttt{CGRect}.

Dans ce chapitre, des additions à ce type sont présentées. Elles sont définies comme des extensions du type \texttt{CGRect}, et implémentées dans \texttt{extension-CGRect.swift}.




\section{Construction d'un rectangle à partir de deux points}

Cet initialiseur permet de construire un rectangle à partir de deux points ; si les points sont confondus, la taille du rectangle est nulle.


\begin{lstlisting}
extension CGRect {
  init (point p1: CGPoint, point p2: CGPoint) {
    origin = CGPoint (x: min (p1.x, p2.x), y: min (p1.y, p2.y))
    size = CGSize (width: abs (p1.x - p2.x), height: abs (p1.y - p2.y))
  }
}
\end{lstlisting}


%!TEX encoding = UTF-8 Unicode
%!TEX root = ../algos-geometriques.tex


\chapter{Cercle}

Un cercle est caractérisé par son centre et son rayon. C'est un type qui n'existe pas en Cocoa, il est défini dans \texttt{ElCanari} par une structure non mutable. Ce type est implémenté dans \texttt{CanariCircle.swift}. Il est simplement construit à partir d'un point et d'un rayon.

\begin{lstlisting}
struct CanariCircle {
  let center : CGPoint
  let radius : CGFloat

  init (center : CGPoint, radius : CGFloat) {
    self.center = center
    self.radius = radius
  }
}
\end{lstlisting}





\section{Intersection entre deux cercles}

Élémentaire, il y a intersection si la distance entre les centres est inférieure ou égale à la somme des rayons.


\begin{lstlisting}
  func intersects (circle : CanariCircle) -> Bool {
    let d = CGPoint.distance (self.center, circle.center)
    return d <= (self.radius + circle.radius)
  }
\end{lstlisting}






\section{Intersection avec un segment}

L'intersection entre un cercle et un segment est plus compliquée, il y a plusieurs tests à faire :
\begin{itemize}
\item d'abord, on teste si la distance entre le centre du cercle et les extrémités du segment ; si l'une de ces distances est inférieure au rayon du cercle, il y a intersection ;
\item ensuite, on calcule les coordonnées du centre du cercle dans le repère dont l'origine est le centre du segment, et l'axe des abscisses la direction du segment ; il y a intersection si et seulement si :
  \begin{itemize}
  \item la valeur absolue de l'ordonnée du centre est inférieure au rayon ;
  \item la valeur absolue de l'abscisse du centre est inférieure à la moitié de la distance entre les deux points.
  \end{itemize}
\end{itemize}

\begin{center}
  \begin{tikzpicture}
    \draw (0, 0) -- (2, 1) ;
    \draw[thick] (0, 0) -- (2, 1) ;
    \draw[fill] (0, 0) circle (2pt) ;
    \draw[fill] (2, 1) circle (2pt) ;
    \draw[left] (0, 0) node {$P_1$} ;
    \draw[above] (2.5, 1.5) node {$C$} ;
    \draw (2.5, 1.5) circle (30pt) ;
    \draw[fill] (2.5, 1.5) circle (2pt) ;
    \draw[right] (2, 1) node {$P_2$} ;
    \begin{scope}[xshift=8cm]
      \draw (0, 0) -- (2, 1) ;
      \draw[thick] (0, 0) -- (2, 1) ;
      \draw[fill] (0, 0) circle (2pt) ;
      \draw[fill] (2, 1) circle (2pt) ;
      \draw[left] (0, 0) node {$P_1$} ;
      \draw[above] (1, 1.5) node {$C$} ;
      \draw (1, 1.5) circle (30pt) ;
      \draw[fill] (1, 1.5) circle (2pt) ;
      \draw[right] (2, 1) node {$P_2$} ;
      \draw (1, 1.5) -- (1.4, .7) ;
    \end{scope}
  \end{tikzpicture}
\end{center}


\begin{lstlisting}
  func intersects (segmentFrom p1 : CGPoint, to p2 : CGPoint) -> Bool {
    var intersects = CGPoint.distance (p1, self.center) <= self.radius
    if !intersects {
      intersects = CGPoint.distance (p2, self.center) <= self.radius
    }
    if !intersects {
      let segmentAngle = CGPoint.angleInRadian (p1, p2)
      let segmentCenter = CGPoint (x: (p1.x + p2.x) / 2.0, y: (p1.y + p2.y) / 2.0)
      let tr = CGAffineTransform (rotationAngle: -segmentAngle)
              .translatedBy (x:-segmentCenter.x, y:-segmentCenter.y)
      let point = self.center.applying (tr)
      intersects = abs (point.y) <= self.radius
      if intersects {
        let segmentLength = CGPoint.distance (p1, p2)
        intersects = abs (point.x) <= (segmentLength * 0.5)
      }
    }
    return intersects
  }
\end{lstlisting}


%!TEX encoding = UTF-8 Unicode
%!TEX root = ../algos-geometriques.tex


\chapter{Rectangle}

Il s'agit de rectangles d'orientation quelconque, ce qui généralise les rectangles « horizontaux » du \refChapterPage{rectangleHorizontal}.

Un rectangle est caractérisé par :
\begin{itemize}
  \item son centre $C$ ;
  \item son angle $\alpha$ avec l'horizontal ;
  \item sa largeur $l$ ;
  \item sa hauteur $h$.
\end{itemize}

\begin{center}
  \begin{tikzpicture}[rotate=19.29]
    \draw[very thick] (0, -.2) rectangle (2, 1.2) ;
    \draw[thin] (1, .5) -- ++ (2, 0) ;
    \draw[thin] (1, .5) -- ++ (2, -0.7) ;
    \draw[fill] (1, 0.5) circle (2pt) ;
    \draw[left] (1, 0.5) node {$C$} ;
    \draw[<->] (-.2, -.2) -- ++ (0, 1.4) ;
    \draw[left] (-.2, .5) node {$h$} ;
    \draw[<->] (0, 1.4) -- ++ (2, 0) ;
    \draw[above] (1, 1.4) node {$l$} ;
    \draw[<-] (2.5, 0.5) arc (0:-30:1) ;
    \draw[right] (2.5, .25) node {$\alpha$} ;
  \end{tikzpicture}
\end{center}

C'est un type qui n'existe pas en Cocoa, il est défini dans \texttt{ElCanari} par une structure non mutable. Ce type est implémenté dans \texttt{CanariRect.swift}.

\begin{lstlisting}
struct CanariRect {
  let center : CGPoint
  let angle : CGFloat // In radians
  let size : CGSize
  
  ...
}
\end{lstlisting}





\section{Construction à partir d'un rectangle horizontal (\texttt{CGRect})}

\begin{lstlisting}
  init (cgrect : CGRect) {
    center = CGPoint (x: NSMidX (cgrect), y: NSMidY (cgrect))
    size = cgrect.size
    angle = 0.0
  }
\end{lstlisting}






\section{Construction à partir de deux points et d'une hauteur}

\begin{center}
  \begin{tikzpicture}[rotate=22.0]
    \draw[very thick] (0, 0) rectangle (2, 1) ;
    \draw[fill] (0, 0.5) circle (2pt) ;
    \draw[fill] (2, 0.5) circle (2pt) ;
    \draw[left] (0, 0.5) node {$P_1$} ;
    \draw[right] (2, 0.5) node {$P_2$} ;
    \draw[<->] (0.4, 0) -- ++ (0, 1) ;
    \draw[right] (0.4, .5) node {$h$} ;
  \end{tikzpicture}
\end{center}


\begin{lstlisting}
  init (from p1 : CGPoint, to p2 : CGPoint, height : CGFloat) {
    center = CGPoint (x: (p1.x + p2.x) * 0.5, y: (p1.y + p2.y) * 0.5)
    size = CGSize (width: CGPoint.distance (p1, p2), height: height)
    angle = CGPoint.angleInRadian (p1, p2)
  }
\end{lstlisting}






\section{Construction à partir du centre, angle et taille}

\begin{lstlisting}
  init (center : CGPoint, size : CGSize, angle : CGFloat) {
    self.center = center
    self.size = size
    self.angle = angle
  }
\end{lstlisting}






\section{Cercle inscrit}

\begin{center}
  \begin{tikzpicture}[rotate=25]
    \draw[very thick] (0, 0) rectangle (2, 1) ;
    \draw[thin] (1, 0.5) circle (0.5cm) ;
  \end{tikzpicture}
\end{center}

\begin{lstlisting}
  func inscribedCircle () -> CanariCircle {
    let radius = min (size.width, size.height) / 2.0
    return CanariCircle (center: self.center, radius: radius)
  }
\end{lstlisting}






\section{Cercle circonscrit}

\begin{center}
  \begin{tikzpicture}[rotate=25]
    \draw[very thick] (0, 0) rectangle (2, 1) ;
    \draw[thin] (1, 0.5) circle (1.118cm) ;
  \end{tikzpicture}
\end{center}

\begin{lstlisting}
  func circumCircle () -> CanariCircle {
    let radius = sqrt (size.width * size.width + size.height * size.height) / 2.0
    return CanariCircle (center: self.center, radius: radius)
  }
\end{lstlisting}







\section{Inclusion d'un point}

Le test d'inclusion d'un point est plus délicat à mettre au point, bien que le code résultant soit court. Le principe est d'effectuer un changement de repère, de façon à obtenir les coordonnées du point testé dans le repère lié au rectangle, qui devient alors un rectangle horizontal. Tester l'appartenance du point devient élémentaire dans ce nouveau repère. 

\begin{center}
  \def\angle{19.29}
  \begin{tikzpicture}
    \begin{scope}[rotate=\angle]
      \begin{scope}[rotate=-\angle]
        \draw[thin,->] (-1, -1) -- ++ (4, 0) node[right] {$x$} ;
        \draw[thin,->] (-1, -1) -- ++ (0, 3) node[above] {$y$} ;
      \end{scope}
      \draw[fill] (1.5, 2.25) circle (2pt) node[right] {$P$} ;
      \draw[very thick] (0, -.2) rectangle ++ (2, 1.4) ;
      \draw[thin] (1, .5) -- ++ (2, 0) ;
      \draw[thin] (1, .5) -- ++ (2, -0.7) ;
      \draw[fill] (1, 0.5) circle (2pt) ;
      \draw[left] (1, 0.5) node {$C$} ;
      \draw[<->] (-.2, -.2) -- ++ (0, 1.4) ;
      \draw[left] (-.2, .5) node {$h$} ;
      \draw[<->] (0, 1.4) -- ++ (2, 0) ;
      \draw[above] (1, 1.4) node {$l$} ;
      \draw[<-] (2.5, 0.5) arc (0:-30:1) ;
      \draw[right] (2.5, .25) node {$\alpha$} ;
    \end{scope}
    \draw (5, 1) node {$\Longrightarrow$} ;
    \begin{scope}[xshift=8cm]
      \draw[thin,->] (0, 0) -- ++ (4, 0) node[right] {$x$} ;
      \draw[thin,->] (0, 0) -- ++ (0, 2) node[above] {$y$} ;
      \draw[very thick] (-1, -.7) rectangle ++ (2, 1.4) ;
      \draw[fill] (0.5, 1.75) circle (2pt) node[right] {$P$} ;
    \end{scope}
  \end{tikzpicture}
\end{center}


\begin{lstlisting}
  func contains (point p : CGPoint) -> Bool {
    let tr = CGAffineTransform (rotationAngle: -angle)
            .translatedBy (x:-center.x, y:-center.y)
    let point = p.applying (tr)
    return (abs (point.x) <= (size.width * 0.5))
        && (abs (point.y) <= (size.height * 0.5))
  }
\end{lstlisting}







\section{Coordonnées des sommets}

La fonction suivante retourne dans un tableau les quatre sommets du rectangle. Le tableau est ordonné, on parcourt les sommets dans le sens trigonométrique.

\begin{lstlisting}
  func vertices () -> [CGPoint] { // Returns the four vertices in counterclock order
    let cosSlash2 = cos (angle) / 2.0
    let sinSlash2 = sin (angle) / 2.0
    let widthCos  = size.width  * cosSlash2
    let widthSin  = size.width  * sinSlash2
    let heightCos = size.height * cosSlash2
    let heightSin = size.height * sinSlash2
    return [
      CGPoint (x: center.x + widthCos - heightSin, y: center.y + widthSin + heightCos),
      CGPoint (x: center.x - widthCos - heightSin, y: center.y - widthSin + heightCos),
      CGPoint (x: center.x - widthCos + heightSin, y: center.y - widthSin - heightCos),
      CGPoint (x: center.x + widthCos + heightSin, y: center.y + widthSin - heightCos)
    ]
  }
\end{lstlisting}







\section{Intersection avec un cercle}

Il y a intersection si :
\begin{itemize}
  \item si le cercle contient le centre du rectangle ;
  \item ou si le rectangle contient le centre du cercle ;
  \item ou, à défaut, si le cercle présente une intersection avec l'un des quatre côtés du rectangle.
\end{itemize}


\begin{lstlisting}
  func intersects (circle : CanariCircle) -> Bool {
  //--- Intersection if circle contains rectangle center
    var intersects = CGPoint.distance (self.center, circle.center) <= circle.radius
  //--- Intersection if rectangle contains circle center
    if !intersects {
      intersects = self.contains (point: circle.center)
    }
  //--- Test intersection between circle and rectangle edge
    if !intersects {
      let vertices = self.vertices ()
      var i = 0
      while !intersects && (i < vertices.count) {
        intersects = circle.intersects (segmentFrom: vertices [i],
                                        to: vertices [(i+1) % vertices.count])
        i += 1
      }
    }
    return intersects
  }
\end{lstlisting}






\section{Intersection avec un autre rectangle}

La méthode qui fait autorité est la méthode dite de \emph{séparation d'axes}. Elle est illustrée dans la video \url{https://www.youtube.com/watch?v=WBy6AveIRRs}. En résumé, il n'y a pas intersection si il existe une droite pour laquelle un rectangle est complètement contenu dans un demi-plan, et l'autre rectangle complètement contenu dans l'autre demi-plan.

\begin{center}
  \begin{tikzpicture}[rotate=20]
    \draw[very thick] (0, 0) rectangle (2, 1) ;
    \begin{scope}[rotate=-30]
      \draw[very thick] (0.5, 2.5) rectangle ++ (2, 1) ;
    \end{scope}
    \draw[thin,dotted] (-.2, 2) -- ++ (5, -2) ;
  \end{tikzpicture}
\end{center}

L'intérêt de la méthode est que l'on n'a pas besoin de construire une telle droite, qui d'ailleurs n'est pas unique dans le cas général.

Il faut commencer par construire les sommets des rectangles.

\begin{center}
  \begin{tikzpicture}[rotate=20]
    \draw[thin,dotted] (2, -1) -- ++ (0, 4) ;
    \draw[very thick] (0, 0) rectangle (2, 1) ;
    \draw[fill] (0, 0) circle (2pt) node[left] {$A$} ;
    \draw[fill] (2, 0) circle (2pt) node[right] {$B$} ;
    \draw[fill] (2, 1) circle (2pt) node[right] {$C$} ;
    \draw[fill] (0, 1) circle (2pt) node[left] {$D$} ;
    \begin{scope}[rotate=-30]
      \draw[very thick] (0.5, 2.5) rectangle ++ (2, 1) ;
      \draw[fill] (0.5, 2.5) circle (2pt) node[left] {$E$} ;
      \draw[fill] (2.5, 2.5) circle (2pt) node[right] {$F$} ;
      \draw[fill] (2.5, 3.5) circle (2pt) node[right] {$G$} ;
      \draw[fill] (0.5, 3.5) circle (2pt) node[left] {$H$} ;
    \end{scope}
  \end{tikzpicture}
\end{center}

Ensuite, pour chaque côté du premier rectangle, on effectue le \emph{test de séparation} : il est positif si les quatre sommets de l'autre rectangle sont « de l'autre côté ». Par exemple, on considère le coté $BC$, qui définit une droite qui partage le plan en deux. Par la fonction \texttt{CGPoint.product}, on calcule la composante verticale de $\overrightarrow{BC} \wedge \overrightarrow{CD}$ : son signe caractérise le demi-plan du premier rectangle. On calcule ensuite la composante verticale de $\overrightarrow{BC} \wedge \overrightarrow{CP}$, pour les quatre sommets $P$ du second rectangle. On obtient un signe contraire pour les sommets $F$, $G$, $H$, mais le même signe pour le sommet $E$, ce qui fait échouer le test de séparation.

Le test de séparation réussit en considérant le segment $EF$ :

\begin{center}
  \begin{tikzpicture}[rotate=20]
    \draw[very thick] (0, 0) rectangle (2, 1) ;
    \draw[fill] (0, 0) circle (2pt) node[left] {$A$} ;
    \draw[fill] (2, 0) circle (2pt) node[right] {$B$} ;
    \draw[fill] (2, 1) circle (2pt) node[right] {$C$} ;
    \draw[fill] (0, 1) circle (2pt) node[left] {$D$} ;
    \begin{scope}[rotate=-30]
      \draw[thin,dotted] (-1.5, 2.5) -- ++ (6, 0) ;
      \draw[very thick] (0.5, 2.5) rectangle ++ (2, 1) ;
      \draw[fill] (0.5, 2.5) circle (2pt) node[left] {$E$} ;
      \draw[fill] (2.5, 2.5) circle (2pt) node[right] {$F$} ;
      \draw[fill] (2.5, 3.5) circle (2pt) node[right] {$G$} ;
      \draw[fill] (0.5, 3.5) circle (2pt) node[left] {$H$} ;
    \end{scope}
  \end{tikzpicture}
\end{center}

Il faut donc effectuer le test de séparation pour chaque côté du premier rectangle {\bf et} chaque côté du second rectangle. Il y a intersection si {\bf tous} les tests de séparation échouent, donc pas d'intersection si un des tests réussit.

Il est indispensable de faire les tests pour les deux rectangles : en effet, ils peuvent échouer pour chaque côté du premier rectangle, alors qu'il n'y a pas intersection. Par exemple dans la figure suivante, les quatre tests d'isolation échouent pour les quatre côtés du rectangle $ABCD$.

\begin{center}
  \begin{tikzpicture}[rotate=20]
    \draw[thin,dotted] (2, -1) -- ++ (0, 4) ;
    \draw[thin,dotted] (0, -.5) -- ++ (0, 4) ;
    \draw[thin,dotted] (-1, 0) -- ++ (6, 0) ;
    \draw[thin,dotted] (-1, 1) -- ++ (6, 0) ;
    \draw[very thick] (0, 0) rectangle (2, 1) ;
    \draw[fill] (0, 0) circle (2pt) node[left] {$A$} ;
    \draw[fill] (2, 0) circle (2pt) node[right] {$B$} ;
    \draw[fill] (2, 1) circle (2pt) node[right] {$C$} ;
    \draw[fill] (0, 1) circle (2pt) node[left] {$D$} ;
    \begin{scope}[rotate=-30]
      \draw[very thick] (0.5, 2.5) rectangle ++ (2, 1) ;
      \draw[fill] (0.5, 2.5) circle (2pt) node[left] {$E$} ;
      \draw[fill] (2.5, 2.5) circle (2pt) node[right] {$F$} ;
      \draw[fill] (2.5, 3.5) circle (2pt) node[right] {$G$} ;
      \draw[fill] (0.5, 3.5) circle (2pt) node[left] {$H$} ;
    \end{scope}
  \end{tikzpicture}
\end{center}


Voici le code de la fonction.

\begin{lstlisting}
  func intersects (rect : CanariRect) -> Bool {
  //--- Method of separating axes (https://www.youtube.com/watch?v=WBy6AveIRRs)
    var intersects = true
    let vertices1 = self.vertices ()
    let vertices2 = rect.vertices ()
    do{
      var i = 0
      while intersects && (i < vertices1.count) {
        let ref = CGPoint.product (vertices1 [i],
                                   vertices1 [(i+1) % vertices1.count],
                                   vertices1 [(i+2) % vertices1.count])
        var outside = true
        var j = 0
        while outside && (j < vertices2.count) {
          let test = CGPoint.product (vertices1 [i],
                                      vertices1 [(i+1) % vertices1.count],
                                      vertices2 [j])
          outside = (ref * test) < 0.0
          j += 1
        }
        intersects = !outside
        i += 1
      }
    }
  //---
    if intersects {
      var i = 0
      while intersects && (i < vertices2.count) {
        let ref = CGPoint.product (vertices2 [i],
                                   vertices2 [(i+1) % vertices2.count],
                                   vertices2 [(i+2) % vertices2.count])
        var outside = true
        var j = 0
        while outside && (j < vertices1.count) {
          let test = CGPoint.product (vertices2 [i],
                                      vertices2 [(i+1) % vertices2.count],
                                      vertices1 [j])
          outside = (ref * test) < 0.0
          j += 1
        }
        intersects = !outside
        i += 1
      }
    }
  //---
    return intersects
  }
\end{lstlisting}


%!TEX encoding = UTF-8 Unicode
%!TEX root = ../algos-geometriques.tex


\chapter{Oblong}

Un oblong est un rectangle terminé par deux demi-cercles :

\begin{center}
  \begin{tikzpicture}[rotate=17]
    \draw[very thick] (0, 0) -- ++ (2, 0) ;
    \draw[very thick] (0, 1) -- ++ (2, 0) ;
    \draw[very thick] (2, 0) arc (-90:90:.5) ;
    \draw[very thick] (0, 0) arc (90:-90:-.5) ;
    \draw[thin, <->] (0.4, 0) -- ++ (0, 1) node[midway, right] {$h$} ;
  \end{tikzpicture}
\end{center}

Un oblong est défini par deux points et la hauteur de la partie centrale, qui est aussi le diamètre des cercles d'extrémité :

\begin{center}
  \begin{tikzpicture}[rotate=17]
    \draw[very thick] (0, 0) -- ++ (2, 0) ;
    \draw[very thick] (0, 1) -- ++ (2, 0) ;
    \draw[very thick] (2, 0) arc (-90:90:.5) ;
    \draw[very thick] (0, 0) arc (90:-90:-.5) ;
    \draw[fill] (0, 0.5) circle (2pt) node[right] {$P_1$} ;
    \draw[fill] (2, 0.5) circle (2pt) node[left] {$P_2$} ;
    \draw[thin, <->] (0.7, 0) -- ++ (0, 1) node[midway, right] {$h$} ;
    \draw[thin, <->] (2.05, 0.5) -- ++ (0.45, 0) node[right] {$h/2$} ;
    \draw[thin, <->] (-0.05, 0.5) -- ++ (-0.45, 0) node[left] {$h/2$} ;
  \end{tikzpicture}
\end{center}


\begin{lstlisting}
struct CanariOblong {
  let p1 : CGPoint
  let p2 : CGPoint
  let height : CGFloat

  init (from p1 : CGPoint, to p2 : CGPoint, height : CGFloat) {
    self.p1 = p1
    self.p2 = p2
    self.height = height
  }
  ...
}
\end{lstlisting}




\section{Point dans un oblong}

Il suffit de tester successivement si :
\begin{itemize}
  \item le point est dans le cercle de centre $P_1$ ;
  \item le point est dans le cercle de centre $P_2$ ;
  \item le point est dans le rectangle formé par la partie centrale ;
\end{itemize}

\begin{lstlisting}
  func contains (point p : CGPoint) -> Bool {
  //--- p inside P1 circle
    var inside = CGPoint.distance (self.p1, p) <= (height / 2.0)
  //--- p inside P2 circle
    if !inside {
      inside = CGPoint.distance (self.p2, p) <= (height / 2.0)
    }
  //--- p inside rectangle
    if !inside {
      let r = CanariRect (from: self.p1, to: self.p2, height: self.height)
      inside = r.contains (point: p)
    }
    return inside
  }
\end{lstlisting}










\section{Intersection avec un autre rectangle}

Il suffit de tester successivement si :
\begin{itemize}
  \item l'autre rectangle intersecte le cercle de centre $P_1$ ;
  \item l'autre rectangle intersecte le cercle de centre $P_2$ ;
  \item l'autre rectangle intersecte le rectangle formé par la partie centrale ;
\end{itemize}

\begin{lstlisting}
  func intersects (rect : CanariRect) -> Bool {
  //--- rect intersects P1 circle
    let c1 = CanariCircle (center: self.p1, radius: self.height / 2.0)
    var intersects = rect.intersects (circle: c1)
  //--- rect intersects P2 circle
    if !intersects {
      let c2 = CanariCircle (center: self.p2, radius: self.height / 2.0)
      intersects = rect.intersects (circle: c2)
    }
  //--- rect intersects rectangle
    if !intersects {
      let r = CanariRect (from: self.p1, to: self.p2, height: self.height)
      intersects = rect.intersects (rect: r)
    }
    return intersects
  }
\end{lstlisting}






\section{Dessiner}

Pour dessiner un oblong, le plus simple est de tracer le segment $P_1P_2$, avec l'épaisseur \texttt{height}. En précisant la terminaison \texttt{kCALineCapRound}, les deux demi-disques sont ajoutés au tracé. Si les points sont confondus, le tracé résulte en un disque.

\begin{lstlisting}
  func shape () -> CAShapeLayer {
    let mutablePath = CGMutablePath ()
    mutablePath.move (to: self.p1)
    mutablePath.addLine (to: self.p2)
    let newLayer = CAShapeLayer ()
    newLayer.path = mutablePath
    newLayer.lineWidth = self.height
    newLayer.lineCap = kCALineCapRound
    return newLayer
  }
\end{lstlisting}


Pour le tracé effectif, il faut préciser sa couleur, par exemple :
\begin{lstlisting}
    let oblong = CanariOblong (...)
    let shapeLayer = oblong.shape ()
    shapeLayer.strokeColor = NSColor.black.cgColor
\end{lstlisting}








\section{Point dans un oblong, autre méthode}

{\bf Cette technique n'est pas implémentée dans ElCanari.}

\begin{center}
  \begin{tikzpicture}
    \draw[line width=8mm, draw=lightgray, line cap=round] (0, 0) -- (2, 1) ;
%    \draw[thick] (0, 0) -- (2, 1) ;
    \draw[fill] (0, 0) circle (2pt) ;
    \draw[fill] (2, 1) circle (2pt) ;
    \draw[left] (-.3, 0) node {$P_1$} ;
    \draw[above] (1, 1.5) node {$P$} ;
    \draw[fill] (1, 1.5) circle (2pt) ;
    \draw[right] (2.3, 1) node {$P_2$} ;
    \draw[<->] (0.5, .7) -- ++ (0.36, -0.72) node[midway, right] {$e$} ;
  \end{tikzpicture}
\end{center}

Il y a un cas particulier si les deux points sont confondus : il suffit alors de calculer la distance entre $P$ et le point $P_1$ (ou $P_2$), et de la comparer avec $e/2$ ; ce n'est pas vraiment un cas particulier cas le cas général commence par tester si point $P$ dans le disque autour des extrémités.

Pour le cas général, on effectue trois tests :
\begin{itemize}
 \item point $P$ dans le disque autour de $P_1$ : calcul de la distance entre $P$ et $P_1$, et comparaison avec $e/2$ ;
 \item point $P$ dans le disque autour de $P_2$ : calcul de la distance entre $P$ et $P_2$, et comparaison avec $e/2$ ;
 \item point $P$ dans la partie centrale : c'est le plus compliqué, et est présenté ci-après. 
\end{itemize}

On considère le point $H (H.x, H.y)$, projection de $P$ sur $P_1P_2$.

\begin{center}
  \begin{tikzpicture}
    \draw[line width=8mm, draw=lightgray, line cap=butt] (0, 0) -- (2, 1) ;
    \draw[thick] (0, 0) -- (2, 1) ;
    \draw[fill] (0, 0) circle (2pt) ;
    \draw[fill] (2, 1) circle (2pt) ;
    \draw[left] (-.3, 0) node {$P_1$} ;
    \draw[above] (1, 1.5) node {$P$} ;
    \draw[fill] (1, 1.5) circle (2pt) ;
    \draw[right] (2.3, 1) node {$P_2$} ;
    \draw (1, 1.5) -- (1.4, .7) ;
    \draw[fill] (1.4, .7) circle (2pt) ;
    \draw[below] (1.4, .7) node {$H$} ;
  \end{tikzpicture}
\end{center}

Le point $P$ est dans la partie centrale si et seulement si :
\begin{itemize}
  \item le point $H$ est entre $P_1$ et $P_2$ ;
  \item la distance $PH$ est inférieure à $e/2$.
\end{itemize}

Nous allons calculer les coordonnées de $H$, que l'on écrit sous la forme :

\begin{equation*}
h.x = \frac{p_1.x + p_2.x}{2} - \lambda ~ \frac{p_1.x - p_2.x}{2}\text{~et~}
h.y = \frac{p_1.y + p_2.y}{2} - \lambda ~ \frac{p_1.y - p_2.y}{2}
\end{equation*}

Ceci assure que $H$ est sur la droite $P_1P_2$ ; si $|\lambda|\leqslant1$, $H$ est entre $P_1$ et $P_2$. Pour calculer $\lambda$, on va écrire que $\overrightarrow{PH}$ et $\overrightarrow{P_1P_2}$ sont orthogonaux.

Ainsi :

\begin{equation*}
  \begin{array}{r|l}
    \overrightarrow{HP}
  &
    \begin{array}{l}
      \displaystyle\frac{p_1.x + p_2.x}{2} - \lambda ~ \frac{p_1.x - p_2.x}{2} - p.x
    \\
      \displaystyle\frac{p_1.y + p_2.y}{2} - \lambda ~ \frac{p_1.y - p_2.y}{2} - p.y
    \end{array}
  \end{array}
\end{equation*}

\begin{equation*}
  \begin{array}{r|l}
    \overrightarrow{P_2P_1}
  &
    \begin{array}{l}
       p_1.x - p_2.x
    \\
       p_1.y - p_2.y
    \end{array}
  \end{array}
\end{equation*}

Pour que ces deux vecteurs soient orthogonaux :

\begin{equation*}
 (\displaystyle\frac{p_1.x + p_2.x}{2} - \lambda ~ \frac{p_1.x - p_2.x}{2} - p.x) ~ (p_1.x - p_2.x) = 
 (\displaystyle\frac{p_1.y + p_2.y}{2} - \lambda ~ \frac{p_1.y - p_2.y}{2} - p.y) ~ (p_1.y - p_2.y)
\end{equation*}

D'où :

\begin{equation*}
\lambda = \frac{(p_1.x + p_2.x - 2~p.x)(p_1.x - p_2.x) + (p_1.y + p_2.y - 2~p.y)(p_1.y - p_2.y)}{(p_1.x - p_2.x)^2 + (p_1.y - p_2.y)^2} = \frac{N}{D}
\end{equation*}


\begin{lstlisting}
func segment (from p1 : CGPoint,
              to p2 : CGPoint,
              halfWidth : CGFloat,
              contains p : CGPoint) -> Bool {
//--- Near First point ?
  var contains = p.distanceTo (point: CGPoint (x: p1.x, y: p1.y)) < halfWidth
//--- Near Second point ?
  if !contains {
    contains = p.distanceTo (point: CGPoint (x: p2.x, y: p2.y)) < halfWidth
  }
//--- In segment ?
  if !contains && (( p1.x != p2.x) || (p1.y != p2.y)) {
    let dx = p1.x - p2.x
    let dy = p1.y - p2.y
    let N = (p1.x + p2.x - 2.0 * p.x) * dx + (p1.y + p2.y - 2.0 * p.y) * dy
    let D = dx * dx + dy * dy
    let lambda = N / D
    contains = abs (lambda) < 1.0
    if contains {
      let hx = (p1.x + p2.x) * 0.5 - lambda * dx * 0.5
      let hy = (p1.y + p2.y) * 0.5 - lambda * dy * 0.5
      contains = p.distanceTo (point: CGPoint (x: hx, y: hy)) < halfWidth
    }
  }
  return contains
}
\end{lstlisting}


%!TEX encoding = UTF-8 Unicode
%!TEX root = ../algos-geometriques.tex


\chapter{Tests d'isolation}

Ces algorithmes sont utilisés pour tester l'isolation entre deux les pistes, les pads, les vias…





Il s'agit de rectangles d'orientation quelconque, ce qui généralise les rectangles « horizontaux » du \refChapterPage{rectangleHorizontal}.

\section{Isolation entre un disque et un rectangle}

Le rectangle est défini par :
\begin{itemize}
  \item son centre $(x_R, y_R)$ ;
  \item son angle $\alpha$ avec l'horizontal ;
  \item sa largeur $l$ ;
  \item sa hauteur $h$.
\end{itemize}

Le disque est défini par :
\begin{itemize}
  \item son centre $(x_C, y_C)$ ;
  \item son rayon $r$.
\end{itemize}

La distance d'isolation est $iso$, c'est-à-dire que la disque entre deux points quelqonques du disque et du rectangle doit être supérieure ou égale à $iso$.

\begin{center}
  \begin{tikzpicture}
    \begin{scope}[rotate=19.29]
      \draw[very thick] (0, -.2) rectangle (2, 1.2) ;
      \draw[thin] (1, .5) -- ++ (2, 0) ;
      \draw[thin] (1, .5) -- ++ (2, -0.7) ;
      \draw[fill] (1, 0.5) circle (2pt) ;
      \draw[below] (1, 0.5) node {\small$(x_R, y_R)$} ;
      \draw[<->] (-.2, -.2) -- ++ (0, 1.4) ;
      \draw[left] (-.2, .5) node {$h$} ;
      \draw[<->] (0, 1.4) -- ++ (2, 0) ;
      \draw[above] (1, 1.4) node {$l$} ;
      \draw[<-] (2.5, 0.5) arc (0:-30:1) ;
      \draw[right] (2.5, .25) node {$\alpha$} ;
      \draw[very thick] (-5, 2.5) circle (.5) ;
      \draw[fill] (-5, 2.5) circle (2pt) ;
      \draw[above] (-5, 3) node {\small$(x_C, y_C)$} ;
    \end{scope}
  \end{tikzpicture}
\end{center}

Nous allons effectuer un changement de repère : l'origine du nouveau repère est le centre du rectangle est son orientation celle de la largeur du rectangle (c'est-à-dire une rotation de $-\alpha$). Le centre du cercle a pour coordonnées $(x'_C, y'_C)$ dans ce nouveau repère.

\begin{center}
  \begin{tikzpicture}
    \begin{scope}
      \draw[very thick] (0, -.2) rectangle (2, 1.2) ;
      \draw[fill] (1, 0.5) circle (2pt) ;
      \draw[below] (1, 0.5) node {\small$(0, 0)$} ;
      \draw[<->] (-.2, -.2) -- ++ (0, 1.4) ;
      \draw[left] (-.2, .5) node {$h$} ;
      \draw[<->] (0, 1.4) -- ++ (2, 0) ;
      \draw[above] (1, 1.4) node {$l$} ;
      \draw[very thick] (-5, 2.5) circle (.5) ;
      \draw[fill] (-5, 2.5) circle (2pt) ;
      \draw[above] (-5, 3) node {\small$(x'_C, y'_C)$} ;
    \end{scope}
  \end{tikzpicture}
\end{center}


Le calcul de $(x'_C, y'_C)$ (en fait $(Xcercle, Ycercle)$) est le suivant~:

\begin{lstlisting}
  let rectHalfWidth  = inRectangleSize.width  / 2.0
  let rectHalfHeight = inRectangleSize.height / 2.0
  let Xrelative = inCircleCenter.x - inRectangleCenter.x
  let Yrelative = inCircleCenter.y - inRectangleCenter.y
  let cosAngleRectangle = cos (inRectangleAngleInRadians)
  let sinAngleRectangle = sin (inRectangleAngleInRadians)
  let Xcercle =   Xrelative * cosAngleRectangle + Yrelative * sinAngleRectangle
  let Ycercle = - Xrelative * sinAngleRectangle + Yrelative * cosAngleRectangle
\end{lstlisting}

Comme on s'intéresse uniquement à vérifier que la distance cercle / rectangle est suffisante, on prend la valeur absolue de chaque coordonnée du centre du cercle~: on se ramène uniquement au premier quadrant.

\begin{center}
  \begin{tikzpicture}
    \begin{scope}
      \draw[very thick] (0, -.2) rectangle (2, 1.2) ;
      \draw[fill] (1, 0.5) circle (2pt) ;
      \draw[below] (1, 0.5) node {\small$(0, 0)$} ;
      \draw[<->] (-.2, -.2) -- ++ (0, 1.4) ;
      \draw[left] (-.2, .5) node {$h$} ;
      \draw[<->] (0, 1.4) -- ++ (2, 0) ;
      \draw[above] (1, 1.4) node {$l$} ;
      \draw[very thick] (5, 2.5) circle (.5) ;
      \draw[fill] (5, 2.5) circle (2pt) ;
      \draw[above] (5, 3) node {\small$(absXcercle, absYcercle)$} ;
    \end{scope}
  \end{tikzpicture}
\end{center}


Ainsi : 
\begin{lstlisting}
  let absXcercle = fabs (Xcercle)
  let absYcercle = fabs (Ycercle)
\end{lstlisting}

La distance est suffisante si le centre du cercle se trouve \emph{au dessus} ou \emph{à droite} de la ligne $ABCD$.

\begin{center}
  \begin{tikzpicture}
      \draw[very thick] (0, -.2) rectangle (2, 1.2) ;
      \draw[fill] (1, 0.5) circle (2pt) ;
      \draw[below] (1, 0.5) node {\small$(0, 0)$} ;
      \draw[<->] (-.2, -.2) -- ++ (0, 1.4) ;
      \draw[left] (-.2, .5) node {$h$} ;
      \draw[<->] (0, -.5) -- ++ (2, 0) ;
      \draw[below] (1, -.5) node {$l$} ;
    %--- Cercle
      \draw[very thick] (5, 2.5) circle (.5) ;
      \draw[fill] (5, 2.5) circle (2pt) ;
      \draw[above] (5.5, 2.5) node[right] {\small$(absXcercle, absYcercle)$} ;
      \draw[->] (1, 0.5) -- ++ (5, 0) ;
      \draw[->] (1, 0.5) -- ++ (0, 3) ;
    %--- Ligne ABCD
      \draw (1, 3.1) node {$\bullet$} node[left] {$A$}
         -- ++ (1, 0) node {$\bullet$} node[above] {$B$}
         arc (90:0:1.9) node {$\bullet$} node[right] {$C$}
         -- ++ (0, -.7) node {$\bullet$}  node[below] {$D$} ;
      \draw[dotted] (2, 1.2) node {$\bullet$} node[above right] {$P$}
        rectangle ++ (1.9, 1.9) node {$\bullet$} node[below left] {$Q$} ;
  \end{tikzpicture}
\end{center}

L'ordonnée des points $A$ et $B$ est $ h / 2 + r + iso$. L'abscisse des points $C$ et $D$ est $ l / 2 + r + iso$. $BC$ est un quart de cercle de centre le sommet supérieur droit du rectangle $P$, et de rayon $r + iso$.

L'isolation est respectée si :
\begin{itemize}
  \item le centre du cercle est \emph{suffisament} à droite ;
  \item ou \emph{suffisament} haut ;
  \item ou si il est dans la surface délimitée par $BQC$.
\end{itemize}

Le centre du cercle est \emph{suffisament} à droite~:
\begin{lstlisting}
  var ok = absXcercle >= (l / 2.0 + r + iso)
\end{lstlisting}


Le centre du cercle est \emph{suffisament} haut~:
\begin{lstlisting}
  if !ok {
    ok = absYcercle >= (h / 2.0 + r + iso)
  }
\end{lstlisting}

Le centre du cercle est est dans le rectangle $BPCQ$ et la surface délimitée par $BQC$
\begin{lstlisting}
  if !ok && (absXcercle >= (l / 2.0)) && (absYcercle >= (h / 2.0)) {
    let dx = absXcercle - l / 2.0
    let dy = absYcercle - h / 2.0
    let distance = sqrt (dx * dx + dy * dy)
    ok = distance >= (r + iso)
  }
\end{lstlisting}



%-----------------------------------------------------------------------------------------------------------------------*
%                                                                                                                       *
%   I N D E X                                                                                                           *
%                                                                                                                       *
%-----------------------------------------------------------------------------------------------------------------------*

%\cleardoublepage % Pour commencer a une page impaire
\phantomsection  % Pour faire correctement pointer l'hyperlien dans la table des matières

%--- Ecrire l'index
{\small
\printindex
}

%-----------------------------------------------------------------------------------------------------------------------*


\end{document}